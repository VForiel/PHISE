\documentclass[12pt]{article}
\parindent 0mm
\usepackage{amssymb,amsmath,amsfonts,latexsym,graphicx,amsthm,amstext}
\usepackage{microtype} % Improve typography
\usepackage{lmodern}
\usepackage{longtable}
\usepackage{color}
% \usepackage{scrhack}
\usepackage{placeins}
% \usepackage{lipsum}
\usepackage{lmodern}
%\usepackage{indentfirst}
\usepackage{ulem}
\usepackage{xcolor}
\usepackage{url}
\usepackage{listings}
\usepackage{comment}
\usepackage{empheq}
\usepackage{mathabx}
\usepackage{siunitx}
\usepackage{lscape}
\usepackage{calc} 
\usepackage{nopageno} %pour retirer la numerotation des pages
\usepackage{hyperref} 

\let\OLDthebibliography\thebibliography
\renewcommand\thebibliography[1]{
  \OLDthebibliography{#1}
  \setlength{\parskip}{0pt}
  \setlength{\itemsep}{0pt plus 0.3ex}
}
\usepackage[
	top=1cm, % Top margin
	bottom=1cm, % Bottom margin
	inner=1.6cm, % Inner margin
	outer=1.6cm, % Outer margin
]{geometry}
 			
\pagestyle{empty}

\begin{document}
Journées SF2A 2024
\hfill
S05 Atelier ASHRA: Status \& prospects in Optical Interferometry

\smallskip
\hrule

\bigskip

\begin{center}
\LARGE \bf Tunable Kernel-Nulling interferometry for direct exoplanet detection\rm

\vspace{0.5cm}

\large  \underline{Vincent Foriel}$\,^{1,*}$, \large Frantz Martinache$\,^1$, \large David Mary$\,^1$

\vspace{0.5cm}

\normalsize

$^1$ \textit{Université Côte d’Azur, Observatoire de la Côte d’Azur Nice, CNRS, Laboratoire Lagrange, France}

\vspace{0.3cm}
$^*$E-mail: {\tt vincent.foriel@oca.eu}

\end{center}
\vspace{-0.8cm}
\subsection*{\Large Abstract}

Nulling interferometry is a promising technique for direct detection of exoplanets. However, the performance of current devices is limited by the sensitivity to phase aberrations. This thesis attempts to overcome those challenges by using a four-telescopes nulling interferometer architecture, called Kernel-Nuller\cite{Martinache et al. 2018}, which includes a recombiner that positions the four signals in phase quadrature. This architecture is based on an integrated optical component containing 14 electronically controlled phase shifters, used to correct optical path differences that would be induced by manufacturing defects. The first part of the study consists in the development of an algorithm providing   the delays to be injected into the component to optimize the performance of that device. This   technique is first evaluated via numerical simulations, then in lab. It is then envisaged to leverage the Nuller mode, soon to be installed on the VLTI as part of the ASGARD project, to test this architecture under real conditions of observation. The next step of this study deals with the analysis of the intensity distributions produced at the output of the Kernel-Nuller\cite{Martinache et al. 2018, Cvetojevic et al. 2022} through a series of observations, against which statistical tests and machine learning techniques are applied to detect the presence of exoplanets. The preliminary results of this study are presented in this presentation.

\textbf{\large Keywords:} Interferometry, Exoplanet, Kernel-Nulling, VLTI, ASGARD 

\begin{thebibliography}{60}

\bibitem{Martinache et al. 2018}  Martinache, Frantz, et Michael J. Ireland. "Kernel-Nulling for a Robust Direct Interferometric Detection of Extrasolar Planets". {\it Astronomy \& Astrophysics} \textbf{619} (2018): A87. https://doi.org/10.1051/0004-6361/201832847.


\bibitem{Cvetojevic et al. 2022} Cvetojevic, N. et al. 3-beam self-calibrated Kernel nulling photonic interferometer (2022). Preprint at http://arxiv.org/abs/2206.04977.

\end{thebibliography}

\end{document}

