\documentclass[12pt]{article}
\parindent 0mm
\usepackage{amssymb,amsmath,amsfonts,latexsym,graphicx,amsthm,amstext}
\usepackage{microtype} % Improve typography
\usepackage{lmodern}
\usepackage{longtable}
\usepackage{color}
% \usepackage{scrhack}
\usepackage{placeins}
% \usepackage{lipsum}
\usepackage{lmodern}
%\usepackage{indentfirst}
\usepackage{ulem}
\usepackage{xcolor}
\usepackage{url}
\usepackage{listings}
\usepackage{comment}
\usepackage{empheq}
\usepackage{mathabx}
\usepackage{siunitx}
\usepackage{lscape}
\usepackage{calc} 
\usepackage{nopageno} %pour retirer la numerotation des pages
\usepackage{hyperref} 

\let\OLDthebibliography\thebibliography
\renewcommand\thebibliography[1]{
  \OLDthebibliography{#1}
  \setlength{\parskip}{0pt}
  \setlength{\itemsep}{0pt plus 0.3ex}
}
\usepackage[
	top=1cm, % Top margin
	bottom=1cm, % Bottom margin
	inner=1.6cm, % Inner margin
	outer=1.6cm, % Outer margin
]{geometry}
 			
\pagestyle{empty}

\begin{document}
Journées SF2A 2024
\hfill
S05 Atelier ASHRA: Status \& prospects in Optical Interferometry

\smallskip
\hrule

\bigskip

\begin{center}
\LARGE \bf Tunable Kernel-Nulling interferometry for direct exoplanet detection\rm

\vspace{0.5cm}

\large  \underline{Vincent Foriel}$\,^{1,*}$, \large Frantz Martinache$\,^1$, \large David Mary$\,^1$

\vspace{0.5cm}

\normalsize

$^1$ \textit{Université Côte d’Azur, Observatoire de la Côte d’Azur Nice, CNRS, Laboratoire Lagrange, France}

\vspace{0.3cm}
$^*$E-mail: {\tt vincent.foriel@oca.eu}

\end{center}
\vspace{-0.8cm}
\subsection*{\Large Abstract}

Nulling interferometry is a promising technique for direct exoplanet detection. However, the performance of current devices is limited by their sensitivity to phase aberrations. This thesis attempts to overcome some of these aberrations by using a four-telescope nulling interferometer architecture, called Kernel-Nuller\cite{Chingaipe et al. 2022}, including a recombiner that place the four signals in phase quadrature. This architecture is based on an integrated optical component containing 14 electronically controlled phase shifters, used to correct possible optical path differences induced by manufacturing defects. The first part of this thesis consisted of developing an algorithm capable of giving the delays to be injected into the component to optimize the performance of the device. We tested this technique via numerical simulations, then in the laboratory. We then plan to take advantage of the Nuller mode which will soon be installed on the VLTI as part of the ASGARD project to test this architecture under realistic observation conditions. The second step of this study lies in the analysis of the intensity distributions produced at the output of the Kernel-Nuller\cite{Chingaipe et al. 2022, Cvetojevic et al. 2022} on series of observations through statistical tests and machine learning techniques in order to detect the presence of exoplanets. Here we present the preliminary results of this study.

\textbf{\large Keywords:} Interferometry, Exoplanet, Kernel-Nulling, VLTI, ASGARD 

\begin{thebibliography}{60}

\bibitem{Chingaipe et al. 2022} Chingaipe, P. M., Martinache, F. \& Cvetojevic, N. Four-input photonic kernel-nulling for the VLTI. in {\it Optical and Infrared Interferometry and Imaging VIII} vol. \textbf{12183} 448–457 (SPIE, 2022).

\bibitem{Cvetojevic et al. 2022} Cvetojevic, N. et al. 3-beam self-calibrated Kernel nulling photonic interferometer. Preprint at http://arxiv.org/abs/2206.04977 (2022).

\end{thebibliography}

\end{document}

